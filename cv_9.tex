%%%%%%%%%%%%%%%%%%%%%%%%%%%%%%%%%%%%%%%%%
% Plasmati Graduate CV
% LaTeX Template
% Version 1.0 (24/3/13)
%
% This template has been downloaded from:
% http://www.LaTeXTemplates.com
%
% Original author:
% Alessandro Plasmati (alessandro.plasmati@gmail.com)
%
% License:
% CC BY-NC-SA 3.0 (http://creativecommons.org/licenses/by-nc-sa/3.0/)
%
% Important note:
% This template needs to be compiled with XeLaTeX.
% The main document font is called Fontin and can be downloaded for free
% from here: http://www.exljbris.com/fontin.html
%
%%%%%%%%%%%%%%%%%%%%%%%%%%%%%%%%%%%%%%%%%

%----------------------------------------------------------------------------------------
%	PACKAGES AND OTHER DOCUMENT CONFIGURATIONS
%----------------------------------------------------------------------------------------

\documentclass[a4paper,10pt]{article} % Default font size and paper size

\usepackage{fontspec} % For loading fonts
\defaultfontfeatures{Mapping=tex-text}
\setmainfont[SmallCapsFont = Fontin SmallCaps]{Fontin} % Main document font

\usepackage{xunicode,xltxtra,url,parskip} % Formatting packages

\usepackage[usenames,dvipsnames]{xcolor} % Required for specifying custom colors

\usepackage[big]{layaureo} % Margin formatting of the A4 page, an alternative to layaureo can be \usepackage{fullpage}
% To reduce the height of the top margin uncomment: \addtolength{\voffset}{-1.3cm}

\usepackage{hyperref} % Required for adding links	and customizing them
\definecolor{linkcolour}{rgb}{0,0.2,0.6} % Link color
\hypersetup{colorlinks,breaklinks,urlcolor=linkcolour,linkcolor=linkcolour} % Set link colors throughout the document

\usepackage{titlesec} % Used to customize the \section command

\usepackage{siunitx}
\usepackage[version=3]{mhchem}
\titleformat{\section}{\Large\scshape\raggedright}{}{0em}{}[\titlerule] % Text formatting of sections
\titlespacing{\section}{0pt}{3pt}{3pt} % Spacing around sections

\begin{document}

\pagestyle{empty} % Removes page numbering

\font\fb=''[cmr10]'' % Change the font of the \LaTeX command under the skills section

%----------------------------------------------------------------------------------------
%	NAME AND CONTACT INFORMATION
%----------------------------------------------------------------------------------------

\par{\centering{\Huge Michinari \textsc{Sakai}}\bigskip\par} % Your name

\section{Personal Data}

\begin{tabular}{rl}
\textsc{Place and Date of Birth:} & Los Angeles, USA  | 16 October 1980\\
\textsc{Address:} & 60 N. Nimitz Hwy. \#1107, Honolulu, HI, USA\\
\textsc{Phone:} & +1-808-206-4357\\
\textsc{email:} & \href{mailto:michinar@hawaii.edu}{michinar@hawaii.edu}
\end{tabular}

%----------------------------------------------------------------------------------------
%	EDUCATION
%----------------------------------------------------------------------------------------

\section{Education}

\begin{tabular}{rp{10.3cm}}	
	\textsc{Dec.} 2015 (expected) & Ph.D. in \textsc{Physics},
	\textbf{University of Hawaii}, Manoa\\
	& \small Thesis: ``High Energy Neutrino Analysis in KamLAND and Application
	to Dark Matter Search''\\
	& \small Advisor: Prof. John G. \textsc{Learned}\\

%------------------------------------------------

	\textsc{Aug.} 2005 - \textsc{Aug.} 2007 & Graduate Program in
	\textsc{Mathematics}, \normalsize\textbf{Sun Moon University}, S.~Korea\\
	& \small Advisor: Prof. Doe-Wan \textsc{Kim}\\

%------------------------------------------------

	\textsc{Aug.} 2005 & Dual B.Sc. in \textsc{Physics} and
	\textsc{Mathematics}, \textbf{Sun Moon University}, S.~Korea\\
	& \small Honors: Double Cum Laude\\
	& \small Advisor: Prof. Ki-Won \textsc{Kim}\\

%------------------------------------------------

\end{tabular}

%----------------------------------------------------------------------------------------
%	WORK EXPERIENCE 
%----------------------------------------------------------------------------------------

\section{Work Experience}

\begin{tabular}{r|p{10.4cm}}
\textsc{Aug. 2009 - }\textit{Current} & Research Assistant\\
& \footnotesize{
	KamLAND: Developed directional reconstruction algorithm for
	high-energy neutrinos.
	First ever physics application (dark matter search) of neutrino
	directionality in scintillator experiments.
}\\
& \footnotesize{
	mini-TimeCube: Lead GEANT4 simulation developer for project.
	Examined trade studies for various neutron capture dopants in
	scintillator.
	Contributed to neutrino/neutron directional reconstruction algorithm.
	Conducted background studies for long-lived isotopes produced from
	cosmogenic muons.
}\\
\multicolumn{2}{c}{} \\

%------------------------------------------------

\textsc{Aug. 2007 - May. 2009} & Teaching Assistant\\
& \footnotesize{
	Taught two undergraduate physics mechanics laboratory courses per semester.
	Received positive reviews.
}\\
\multicolumn{2}{c}{} \\

%------------------------------------------------

\textsc{Jan. 2003 - Mar. 2006} & Interpreter and Teacher\\
& \footnotesize{(Mar. 2006) Part time English lecturer for Korean undergraduate
students.}\\
& \footnotesize{(Mar. 2004 - Dec. 2005) Part time contributing reporter and
translator for campus magazine.}\\
& \footnotesize{(Jul. 2004) Spontaneous trilingual interpreter for W-CARP
International Education Conference.}\\
& \footnotesize{(Mar. 2003 - Mar. 2004) Part time translator for magazine Today's World.}\\
\end{tabular}

%----------------------------------------------------------------------------------------
%	SKILLS 
%----------------------------------------------------------------------------------------

\section{Skills}

\begin{tabular}{rl}
	Software/Tools: & \textsc{ROOT}, \textsc{Geant4}, \textsc{Pads}\\
	Programming Languages: & C++, Python, Fortran, Perl, Mathematica, Matlab, Bash, VHDL\\
	Human Languages: & English, Japanese, Korean
\end{tabular}

%----------------------------------------------------------------------------------------
%	SCHOLARSHIPS AND AWARDS
%----------------------------------------------------------------------------------------

\section{Scholarships and Awards}

\begin{tabular}{rl}
	2004 & Award for Outstanding Academic Acheivement, Samsung Corp.\\
	2001, 2002, 2003, 2004 & Undergraduate Achievement Scholarships, Sun Moon
	Univ.\\
	2001 & Ae-Guk Freshman Scholarship, Sun Moon Univ.\\
\end{tabular}

%----------------------------------------------------------------------------------------
%	LANGUAGES
%----------------------------------------------------------------------------------------

%\section{Languages}
%
%\begin{tabular}{rl}
%	\textsc{English}, \textsc{Japanese}, \textsc{Korean}\\
%\end{tabular}
%
%\newpage

%----------------------------------------------------------------------------------------
%	PUBLICATIONS
%----------------------------------------------------------------------------------------

\section{Publications}

\begin{tabular}{rp{11cm}}
	\multicolumn{2}{l}{\textsc{mini-TimeCube}} \\
	2015 (expected) & V.A. Li et al., \textsc{mini-TimeCube}, RSI Invited Review\\
	\multicolumn{2}{c}{} \\
	\multicolumn{2}{l}{\textsc{KamLAND}} \\
	Mar. 2015 & K. Asakura et al., \textsc{Study of electron anti-neutrinos
	associated with gamma-ray bursts using KamLAND}, arXiv:1503.02137v1\\
	Feb. 2015 & T.I. Banks et al., \textsc{A compact ultra-clean system for
	deploying radioactive sources inside the KamLAND detector},
	10.1016/j.nima.2014.09.068\\
	Jan. 2015 & C. Lane et al., \textsc{A new type of Neutrino Detector for
	Sterile Neutrino Search at Nuclear Reactors and Nuclear Nonproliferation
	Applications}, arXiv:1501.06935v1\\
	May 2014 & A. Gando et al., \textsc{7Be Solar Neutrino Measurement with
	KamLAND}, arXiv:1405.6190v1\\
	Aug. 2011 & S. Abe et al., \textsc{Measurement of the 8B Solar Neutrino
	Flux with the KamLAND Liquid Scintillator Detector},
	10.1103/PhysRevC.84.035804\\
	Aug. 2011 & J. Kumar, J.G. Learned, M. Sakai, S. Smith,
	\textsc{Dark Matter Detection With Electron Neutrinos in Liquid
	Scintillation Detectors}, Phys.Rev. D84 (2011) 036007\\
\end{tabular}

%----------------------------------------------------------------------------------------
%	POSTERS AND TALKS
%----------------------------------------------------------------------------------------

\section{Posters and Talks}
\begin{tabular}{rp{11cm}}
	Aug. 2010 & Talk at AAP 2010, Sendai, Japan: mini-TimeCube: A Portable
	Directional Neutrino Detector\\
	Jun. 2012 & Poster at Neutrino 2012, Kyoto, Japan: Indirect Dark-Matter
	Detection Through KamLAND\\
\end{tabular}

%----------------------------------------------------------------------------------------
%	STATEMENT OF RESEARCH INTERESTS AND EXPERIENCE
%----------------------------------------------------------------------------------------

\section{Statement of Research Interests and Experience}
My main interest lies in directional neutrino reconstruction and its
applications such as indirect dark matter searches, directional geo-neutrino
measurements, and anti-nuclear proliferation techniques that involve locating
the position of the source.

I have been involved with three projects during my graduate studies at
University of Hawaii with Prof. John Learned; the \SI{1}{\kilo\tonne} liquid
scintillator neutrino experiment KamLAND in Japan, a portable
\SI{2.2}{\liter} plastic scintillator neutrino experiment called the
mini-TimeCube, and a third related to scintillator R\&D for a
future \SI{10}{\kilo\tonne}-scale deep-sea based neutrino detector HanoHano.

My work in KamLAND has involved developing directional event reconstruction
methods for high-energy $\sim$\si{\giga\electronvolt} scale neutrinos and
applying this to conduct an indirect dark matter search by looking at neutrinos
from the Earth's core.
Studies done with Monte-Carlo suggest that the accuracy of reconstructing the
neutrino direction using this method is better than that of the
water-Cherenkov detectors by $\sim$\SI{10}{\degree} for energies
$\sim$\SI{1}{\giga\electronvolt} and greater.
This method is now being tested against events spilling into KamLAND from the
T2K neutrino beam-line and the initial results are consistent with what is
expected.
I believe this is a first ever physics application to neutrino directionality in
a scintillator experiment.

In addition, I have worked as the lead GEANT4 simulation designer for the
mini-TimeCube collaboration to conduct case studies for optimizing the detector
design and test candidate neutron capture doping elements in plastic
scintillator.
These studies were used during construction of the detector, and to develop
directional algorithms that are now being tested in analysis of neutrons from
test sources as well as neutrinos from nuclear reactors.
I have also conducted simulation studies for long-lived background isotopes such
as \ce{^{8}He} and \ce{^{9}Li} produced by cosmogenic muons.
These backgrounds are extremely difficult to tag due to their long life-time
($>\sim$\si{\second} scale) and long travel distances.
The studies have been vital to the project.
Working with the mini-TimeCube project has further involved fabricating test
boards using the Pads PCB design suit and contributing to the FPGA firmware for
the readout electronics.

Finally, my work in scintillator R\&D for HanoHano has been designing and
building apparatus using CAD for measuring light output of LAB based liquid
scintillators when put in large electromagnetic potential gradiants as well as
testing their light transmissivity when placed under extreme pressures (such as
those found in deep-sea environments).

%----------------------------------------------------------------------------------------

\end{document}
